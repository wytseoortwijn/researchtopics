\chapter{Detailed Approach}
This section provides research questions for this problem, as well as the project goals, and the activities that will be performed during the project.

\section{Research Questions}
This research project will attempt to implement BDD operations for multicore clusters. Implementing these operations is not trivial, since ideally both CPU and memory capacity of all participating clusters has to be utilized as efficiently as possible. A perfect solution would split work perfectly across the participating processes such that every processing element only needs to access its own local memory. This however cannot easily be achieved, if possible at all. So the first step in the research project is to find out how to deal with this problem. This raises the following three research questions.

\begin{enumerate}
	\item[Q.1]{\textit{How can BDDs be spread over all participating clusters so that the memory of all clusters is used efficiently?}}
	\item[Q.2]{\textit{How can work be partitioned such that every participating process can work as efficiently as possible?}}
	\item[Q.3]{\textit{How can load-balancing be implemented such that the number of network operations is minimized?}}
\end{enumerate}

At the end of the Research Topics phase an answer to these three research questions will be given. The goal of the reseach project is to create an implementation based on the ideas from the Research Topics phase. Several benchmarks can be applied to the implementation in order to determine its efficiency. The following research questions need to be answered after the experimental phase.

\begin{enumerate}
	\item[Q.4]{\textit{How well does the implementation scale across the number of participating CPU cores?}}
	\item[Q.5]{\textit{How well does the implementation scale across the number of participating machines?}}
	\item[Q.6]{\textit{How can the scalability of the implementation be improved?}}
\end{enumerate}

If time permits the result of research question \textit{Q.6} can be used to make the implementation more efficient. In that case, the implementation will be improved and the measurement plan will be re-applied to the implementation. Also the answers given to the research questions can be adjusted. If time runs short, the answer of \textit{Q.6} will be marked as future work.

At the end of the research project a conclusion will be given based on the experimental results. In this conclusion the following main research question will be answered.

\begin{enumerate}
	\item[Q.7]{\textit{Can BDD operations be implemented to perform efficiently on a cluster of multicore machines?}}
\end{enumerate}

Note that \textit{performing efficiently} means utilizing the available hardware resources efficiently. Efficiency is something that can be measured, so a relevant answer to this question can be given in the conclusion of the research project.

\section{Goals and Objectives}
The primary goal of this research project is to create an efficient implementation for BDD operations on multicore clusters that scales well across the number of processes and the number of participating clusters. 

The second goal is to create a paper out of the results obtained from the research. The relevance of these results has already been discussed in the motivational part of this report.

The third goal, which is a personal goal, is to try to obtain a PhD funding during or after the graduation project in order to continue working on heterogeneous algorithms.

\section{Methods and Techniques}
The implementation will be programmed in the programming language C. The implementation of the distributed hash table will use MVAPICH2-X 2.1a in combination with OpenSHMEM. The implementation of the work-stealing protcol will be an adjusted version of Lace. The implementation of the BDD operations will be an adjusted version of the operations implemented in the multicore BDD package Sylvan.

In the experimentation phases of the project, the models from the \emph{BEEM} database are used as well as the microbenchmark \emph{YCSB}.

\section{Assumptions}
Unfinished...

\section{Risks}
Unfinished...

\section{Activities}
The following activities will be performed during the research project.

\begin{itemize}
	\item Performing research to find out how BDD operations can efficiently be implemented on a cluster of multicore machines.
	\item Implementing an RDMA-enabled lockless distributed hash table.
	\item Microbenchmarking the distributed hash table.
	\item Implementing the BDD operations on top of this hash table.
	\item Implementing a work-stealing protocol that works on a cluster of multicore machines.
	\item Benchmarking the work-stealing protocol.
	\item Benchmarking the implementation of the BDD operations and reflecting on them.
	\item Presenting the results obtained from the experimental phase in a lunch meeting presentation.
	\item If time permits, implementing some of the possible improvements found in the experimental phase.
	\item Finishing the master thesis.
	\item Performing the master defence.
	\item Writing the paper containing implementation details and experimental results.
\end{itemize}

The next section will contain a project planning in which all the activities given above are planned.

\section{Deliverables}
The research project will result in the following products.

\begin{enumerate}
	\item A literature study in which related work is discussed an in which research questions \textit{Q.1}, \textit{Q.2}, and \textit{Q.3} are answered.
	\item A, RDMA-based distributed hash table that scales well across the available memory of all participating nodes.
	\item A report in which the performance of the distributed hash table is evaluated. This report is the result of microbenchmarking the hash table.
	\item An implementation of the BDD operations that makes use of the distributed hash table.
	\item An report that evaluates the performance of these BDD operations. This report is the result of the experimental phase of the research project.
	\item A conclusion of the experimental phase.
	\item A lunch meeting presentation in which the outcome of the experimental phase, as well as the project itself is presented to the other members of the research group.
	\item A master thesis covering all elements discussed above in this summation.
	\item A master thesis presentation.
	\item A paper in which the implementation and the obtained results are given.
\end{enumerate}